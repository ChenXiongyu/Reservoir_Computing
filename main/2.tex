\documentclass[notitlepage,cs4size,punct,oneside]{ctexrep}

\usepackage[a4paper,hmargin={3.18cm,3.18cm},vmargin={2.54cm, 2.54cm}]{geometry}
\usepackage{amsmath,amssymb,amsthm}
\usepackage{titlesec}
\usepackage{color}
\usepackage[runin]{abstract}
\usepackage[pdfborder={0 0 0},colorlinks=False,CJKbookmarks=true]{hyperref}
\usepackage{fancyhdr}
\usepackage{setspace}
\usepackage{graphicx}

\pagestyle{plain}
\setstretch{1.41} 

\setlength{\absleftindent}{1.5cm} \setlength{\absrightindent}{1.5cm}
\setlength{\abstitleskip}{-\parindent}
\setlength{\absparindent}{0cm}

\numberwithin{equation}{chapter}

\newtheoremstyle{mystyle}{3pt}{3pt}{\kaishu}{0cm}{\heiti}{}{1em}{}
\theoremstyle{mystyle}
\newtheorem{definition}{\hspace{2em}定义}[chapter]
\newtheorem{theorem}[definition]{\hspace{2em}定理}
\newtheorem{axiom}[definition]{\hspace{2em}公理}
\newtheorem{lemma}[definition]{\hspace{2em}引理}
\newtheorem{proposition}[definition]{\hspace{2em}命题}
\newtheorem{corollary}[definition]{\hspace{2em}推论}
\newtheorem{remark}{\hspace{2em}注}[chapter]
\def\theequation{\arabic{chapter}.\arabic{equation}}
\def\thedefinition{\arabic{chapter}.\arabic{definition}.}

\title{{\zihao{-1}\heiti{} 从储备池计算的可塑性到耦合同步现象}}
\author{陈熊宇\\学号:18300180076\\专业:数学与应用数学\\指导老师:林伟}

\date{}

%%%%%%%%%%%%%%%%%%%%%%%%%%%%%%%%%%%%%%%%%%%%%%%%%%%%%%%%%%%%%%%%%%%%%%%%%%%%%%%%%%

\begin{document}

\CTEXsetup[nameformat={\zihao{3}\heiti},%
           titleformat={\zihao{3}},%
           beforeskip={0.8cm},afterskip={1.2cm}]{chapter}
\CTEXsetup[format={\centering\zihao{-3}\heiti},%
           name={第,节},number={\arabic{section}}]{section}
\CTEXsetup[format={\zihao{-4}\heiti},%
           number={\arabic{section}.\arabic{subsection}.},%
           beforeskip={0.4cm},afterskip={0.4cm}]{subsection}
\CTEXoptions[contentsname={\heiti{目\ \ \ \ \ \ 录}}]
\CTEXoptions[abstractname={摘要:}]
\CTEXoptions[bibname={\heiti 参考文献}]

\renewcommand{\thepage}{\roman{page}}

\setcounter{page}{0}

\tableofcontents

%%%%%%%%%%%%%%%%%%%%%%%%%%%%%%%%%%%%%%%%%%%%%%%%%%%%%%%%%%%%%%%%%%%%%%%%%%%%%%%%%%
% \maketitle\renewcommand{\thepage}{\arabic{page}}
\clearpage

\noindent{国内图书分类号: O235}
\hfill{学校代码: 10246}

\bigskip

\bigskip
\bigskip
\bigskip
{\centering{\zihao{-1}\heiti 从储备池计算的可塑性到耦合同步现象}}
\bigskip

\bigskip
\begin{center}
{\zihao{3}\ 陈熊宇}

\bigskip

{\zihao{3}\ 学号:18300180076}

\bigskip

{\zihao{3}\ 专业:数学与应用数学}

\bigskip

{\zihao{3}\ 指导老师:林\ \ 伟\ 教授}
\end{center}
\thispagestyle{empty}\setcounter{page}{0}


% \renewcommand{\abstractname}{摘要}
% \begin{abstract}
\clearpage
{\heiti 摘要\ \ }预测复杂的动力系统的演化是人们关注的焦点, 由于其本身机制复杂, 且受制于观测条件, 
再加上其对数值扰动极其敏感, 预测其未来发展一直是一个难题. 
近年来, 随着计算机计算能力的提升, 基于大量数据的机器学习算法展现出较强的能力. 
本文研究在储备池计算框架下的预测非线性动力系统的问题, 
并研究该问题下储备池计算框架的可塑性, 最后在建立的框架上研究储备池耦合同步的现象. %

本文首先介绍储备池计算的框架, 包括其训练阶段, 预测阶段的计算方法, 
再推广储备池计算一般框架中的激活函数和输出函数形式, 分析不同的函数形式在具体问题下的表现, 
展现储备池计算本身框架的可塑性. 在推广后的框架基础上研究了一种并联的储备池计算框架, 
分析并联的储备池的预测能力相比于单一的储备池能否起到提升作用. %

本文其次将运用设计的最优储备池计算框架, 研究多个储备池之间耦合同步的现象, 
并且在有噪声, 以及不同耦合强度的情况下分析其耦合效果. %

最后考虑到动力系统可能具备不完全可观测条件, 因此通过观测部分变量, 
研究储备池预测全局变量的能力. 结合储备池耦合同步的概念, 分析在该条件下是否也能产生
类似的耦合同步现象, 从而进一步加深对于储备池计算预测能力的认识, 和对于储备池
计算框架下非线性动力系统耦合同步现象的理解. 

{\heiti 关键字: } 储备池计算, 可塑性, 耦合同步.
% \end{abstract}

\clearpage

% \renewcommand{\abstractname}{Abstract}
% \begin{abstract}
{\textbf{Abstract}\ \ }Predicting the evolution of complex dynamical systems 
has been the focus of research. Due to its complex mechanism, vulnerability 
to observational conditions, and its extremely sensitivity to numerical 
perturbations, predicting its future development has always been a difficult 
problem. In recent years, with the improvement of machine computing power, 
machine learning algorithms based on large amounts of data have shown strong 
capabilities. This thesis studies the problem of predicting nonlinear dynamical 
systems in the framework of reservoir computing, and study the plasticity of 
the reservoir computing framework under this problem. Finally it studies the 
phenomenon of reservoir coupling synchronization on the established framework.

This thesis first introduces the framework of the reservoir computing, 
including its training stage and the prediction stage. Then it generalizes 
the form of activation function and output function, 
and analyzes the performance of different function forms under specific problems,
which demonstrates the plasticity of the framework of reservoir computing. 
Based on the extended framework, a parallel reservoir computing framework is 
analyzed whether the predictive ability can be improved compared to a 
single reservoir.

Then this thesis utilizes the designed optimal reservoir computing framework 
to study the phenomenon of coupling synchronization between multiple reservoirs.
In the presence of noise and different coupling strengths, this thesis compares 
their different coupling effect.

Finally, given the dynamic system may have incomplete observable conditions, 
by observing part of the variables, the reservoir is trained to predict the global 
variables. Combined with reservoir coupling, this thesis analyzes whether similar 
coupling synchronization phenomenon can also occur under this condition, 
which further deepens the understanding of reservoir computing and coupling 
synchronization of nonlinear dynamical systems under reservoir computing framework. 

{\textbf{Keywords:}} Reservoir Computing, Plasticity, Coupling Synchronization. 
% \end{abstract}

%%%%%%%%%%%%%%%%%%%%%%%%%%%%%%%%%%%%%%%%%%%%%%%%%%%%%%%%%%%%%%%%%%%%%%%%%%%%%%%%%%

\chapter{绪论}

\section{研究背景及意义}

机器学习作为人工智能其中的一个分支, 关键在于其具备学习能力. 
近年来, 随着机器计算能力的不断发展, 数值预测模拟模型的实际意义大大增加,
机器学习依赖大量数据的特性在时代背景下大放光彩, 推进了一些公开的问题的进展. 

一直以来, 复杂的动力系统的演化规律是人们关注的焦点. 
在实验研究中, 非线性动力系统常常扮演了重要的角色. 
很多时间序列, 包括化学反应系统, 生物神经系统等等都与非线性动力系统有关,
这些系统只能在一定时间内进行测量, 而且目前对于其数学表达形式往往不清楚, 
或者已经有类似化学主方程的模型, 但由于其计算量过大, 仍然无法从取得确切的数值解. 
因此在不知道数学表达形式的情况下, 预测这些系统未来如何演化成为了一个非常有价值的问题.
机器学习对此给出了一定的回答, 例如利用神经网络进行天气预测\cite{VKMF}等等. 

机器学习突破了具体模型的限制, 拟合未知函数的效果具有显著优势, 
其中神经网络的作用不可忽视. 神经网络最初希望通过模拟生物神经元的电位变化, 
实现类似生物神经网络的学习, 随着计算机神经网络的设计越发复杂, 机器学习神经网络的
方法突破了模拟生物神经网络的范畴, 逐渐成为一个成熟的模型. 

储备池计算(Reservoir Computing)是一种由数据驱动的神经网络, 
通过储备池(Reservoir)中大量神经元的演化, 学习模拟各种动力系统. 
储备池相较于其他神经网络, 优势在于其内部构造以及训练过程的简单性, 
这一特点使得训练储备池神经元的速度提升, 并且赋予其在神经网络物理实现上的灵活性\cite{YTJK}, 
尤其适合边缘计算(Edge Computing)需要轻量化的特点\cite{WSJC}.

在这样的背景下, 储备池计算迅速得到发展. 
其中一个研究方向是对于储备池计算可塑性的研究, 
即研究储备池计算在具体问题下的最优网络设置, 包括对于储备池的研究, 训练规则的研究等等, 
从而根据提前设定的最优网络, 对具体应用场景给出事半功倍的效果; 
另一个研究方向是研究储备池实现的同步现象, 储备池的训练过程本质上是将
储备池输出和输入同步的过程\cite{QZHM}, 实现同步之后则可对未来进行预测. 
同步现象常见于各种物理, 化学, 生物系统中, 
例如在耦合的混沌振子中可以实现各种同步现象, 包括完全同步\cite{LPTC}, 相位同步\cite{MRAP}等. 
因此对储备池计算同步现象的研究有利于进一步理解储备池计算背后的理论依据. 


\section{国内外的研究现状}

虽然非线性动力系统作为确定性系统, 但经常其存在混沌现象, 非常依赖初值的准确性, 
这使得混沌现象的预测难度非常大. 早期机器学习需要不断地去调整权重参数进行学习, 
学习过程非常缓慢, 学习效果也一般. 之后液体状态机\cite{WMTN}和回声状态网络\cite{HJ}
被相继提出, 储备池计算就在此基础上被提出, 并给这一难题提供了一种解决方法. 

储备池计算的想法是随机生成神经网络的内部结构, 在固定网络内部结构的基础上, 
只通过训练其输出层的权重来达到学习的目的. 
Schrauwen等人总结了储备池计算的整体框架, 
指出任何高维驱动的动态方式都可以作为储层计算的学习对象, 
使得仅使用线性处理就能解决复杂的任务\cite{BSDV}; 
Pathak等人利用储备池计算对时空混沌系统数据进行了预测, 
证明了储备池计算进行无模型预测的有效性, 
其展示了该模型可以被推广到其他混沌系统的预测中\cite{JPBH}. 
Fan等人将一组时间序列数据输入储备池计算框架中, 
证明了其为各种混沌系统提供任意长的预测范围\cite{JFJJ}. 
可见储备池计算是一种非常灵活, 有效的机器学习框架. 

储备池计算的有效性已经得到充分的说明, 其可塑性也具有很大的发展潜力, 产生了诸多研究成果. 
一个比较直观的是储备池的可塑性, 即改变内部神经元的连接结构, 从而提高预测效果. 
Griffith等人研究了储备池中神经元的连接性, 发现当神经网络相对稀疏, 
单个神经元仅与其他少部分神经元有相互作用时, 
储备池计算有更好的表现, 
稀疏矩阵的计算不仅有利于增加神经元个数, 扩大训练数据规模, 
还有利于进一步提高神经元更新速度\cite{AGAP}; 
还可以对神经元的演化方式进行调整, 
Carrol尝试改变神经元迭代更新的方程, 提供了储备池计算在预测Sprott系统时的一些
超参数设定, 分析其对神经元个数和噪声的依赖性\cite{TLC}; 
储备池计算由于简明的结构, 其可塑性还体现在可以生物神经元的特性引入, 
Morales等人受生物神经元可塑性规则启发, 将其运用到储备池计算中, 
并结合突触和非突触的可塑性规则两种不同形式的神经元可塑性规则, 
提高了原始模型的性能表现\cite{GBMC}; 
另外, 在改变储备池计算网络设计的同时, 可以利用储备池计算的特性完成特定的任务, 
Wendson等人发现了可以利用储备池计算的对称性质, 完成对称性识别任务, 
相比于其他模型具有优势\cite{WASB}. 
更进一步, 类似深度学习的思想, 还可以将多个储备池进行组合得到更加复杂的网络, 例如 
Moon等人总结了分层储备池结构对其性质和性能的影响, 发现将多个储备池串联堆叠, 
得到继承性的储备池网络, 其具有一定的加强作用\cite{JMYW}. 

通过改造储备池结构提高性能的同时, 对于储备池有效性的原因的认识也在进一步加深, 
其中很重要的角度是同步的概念. Fan实现的长期预测本质上实现了混沌系统和储备池神经元之间的同步, 
其刻画了对于时间同步理论的物理模型\cite{JFJJ}; 
Weng等人仅通过一个共同的标量信号也实现了两者之间的同步\cite{TWHY}, 并且
进一步说明其同步的稳健性, 包括对于噪声的鲁棒性\cite{XCTW}. 
而对于本身存在耦合现象的混沌动力系统的预测, Saha等人发现
当仅输入网络的一个振荡节点的单输入数据用于学习时, 储备池计算
能够预测其他节点的动态变化, 包括尖峰和突发\cite{SSAM}. 
除了混沌系统和储备池之间的同步, 多个储备池之间的同步也是研究的对象, 
Hu等人将两个储备池计算耦合,通过调整耦合强度,
也在一定程度上实现了两者之间的同步\cite{WHYZ}.


\section{论文研究思路}

从上述研究现状可以看出, 储备池计算在最优结构设计方面仍有不少发展潜力, 
并且也逐渐引发了耦合同步方面的进一步的理解. 
后文将聚焦于上述两个方向, 第二章从引入储备池计算的框架出发, 
介绍其基本设置, 包括储备池, 激活函数, 输出函数的定义, 
阐释其训练阶段和预测阶段具体方法, 再介绍一种并联储备池网络的框架. 

第三章将研究储备池计算的可塑性, 主要
推广储备池计算中的激活函数, 输出函数, 分析对比神经网络中各种激活函数
放入储备池计算框架后, 在一些具体非线性混沌动力系统预测问题下的表现, 对结果进行
数值模拟, 误差分析, 观察将不同设置的储备池加入到并联储备池网络中能否起到提升作用. 

第四章将关注储备池计算耦合同步的内容, 介绍储备池耦合的框架, 
分析多个储备池之间耦合同步的条件, 考虑耦合中噪声的引入对于同步效果的影响. 

第五章将集中分析在动力系统只具备部分可观性条件下的储备池计算的预测效果, 
以藏本模型(Kuramoto model)为例, 将其部分变量输入到储备池中, 要求预测系统全局变量, 
对各种设定进行数值实验. 在此之上再结合第四章耦合同步的概念, 分析部分变量的同步
能否引发全局变量的同步, 同样对噪声进行鲁棒性分析, 从而进一步理解储备池耦合同步的条件, 
加深对于非线性动力系统同步现象的认识.  

第六章将对本文讨论的储备池计算框架的具体问题进行总结讨论, 分析其优劣势, 
对其中的同步现象进行思考, 并展望未来的储备池计算, 非线性动力系统预测问题的发展. 

%%%%%%%%%%%%%%%%%%%%%%%%%%%%%%%%%%%%%%%%%%%%%%%%%%%%%%%%%%%%%%%%%%%%%%%%%%%%%%%%%%

\chapter{储备池计算介绍}
储备池计算是神经网络的一种框架, 
框架包括输入矩阵, 储备池矩阵, 输出矩阵, 激活函数, 输出函数等等. 
使用包括训练阶段和预测阶段, 其中训练阶段是监督性的, 预测阶段可监督也可以无监督. 
将多个储备池并联就可以得到并联储备池计算网络. 

\section{储备池框架}
储备池框架主要分为三大部分, 第一部分是输入层, 包含需要输入的时序数据$i(t)$, 
输入矩阵$W_{\textmd{in}}$; 第二部分是储备池层, 包含神经元的状态变量$r(t)$, 
储备池矩阵$W_{r}$, 
激活函数$F_{\textmd{act}}$; 第三部分是输出层, 包含输出变量$o(t)$, 
输出矩阵$W_{\textmd{out}}$, 
输出函数$F_{\textmd{out}}$. 如图2.1所示. 

\begin{figure}[htbp]
    \centering
    \includegraphics[scale=0.4]{1.png}
    \caption{储备池框架.}
\end{figure}

假设对于某个时刻$t$来说, 
$i(t)\in\mathbb{R}^{N_i}$, 
$o(t)\in\mathbb{R}^{N_o}$, $r(t)\in\mathbb{R}^{N_r}$, 
那么$W_{\textmd{in}}\in\mathbb{R}^{{N_r}\times{N_i}}$, 
$W_{r}\in\mathbb{R}^{{N_r}\times{N_r}}$,
$W_{\textmd{out}}\in\mathbb{R}^{{N_o}\times{N_r}}$,
其中$N_i$是输入数据的维度, $N_r$是储备池中神经元个数, 
$N_o$是输出数据维度, 没有特殊说明则$N_o=N_r$. 

另外再假设$t\in[0, T]$, $T>0$, 每次时间迭代步长为$\Delta t>0$. 不妨设$T=n\Delta t$. 
因此可以将输入数据, 输出数据按时间进行列的叠加, 得到$[i(0), \dots, i(T)]$和$[o(0), \dots, o(T)]$, 
设这两个拼接矩阵分别为$I\in\mathbb{R}^{{N_i}\times n}$, $O\in\mathbb{R}^{{N_o}\times{n}}$.


\subsection{输入矩阵和储备池矩阵}
假设$W_{\textmd{in}}$中的每个非零元素满足$[-\sigma, \sigma]$区间上的正态分布, 其中$\sigma>0$为参数. 
而且假设$i(t)$每个分量对应相同数目的神经元, 即$W_{\textmd{in}}$每列具有相同数目的非零元素, 
$W_{\textmd{in}}$中所有非零元素个数为$N_r$. 

先假设$W_{r}$中的每个非零元素满足$[0, 1]$区间上的正态分布. 
研究表明储备池矩阵应具有稀疏性\cite{AGAP}, 限制每个神经元只能与固定个数的神经元有联系, 
定义该参数为储备池矩阵的度(degree), 不妨固定为3, 即下个时刻神经元的状态由当前时刻某3个神经元的状态所决定, 
那么$W_{r}$每行具有3个非零元素. 之后还需要对$W_r$的谱半径进行调整, 通过乘以常数使得
其谱半径变为参数$\rho$.

\subsection{激活函数}

在储备池框架中, 激活函数一般设定为双曲正切函数$F_{\textmd{act}}=\tanh(x)$, 本文中将
机器学习框架中其他几个常见的激活函数进行引入, 具体如下:

ReLU函数, 即正部函数, 可写为:
$$
    F_{\textmd{act}} = \frac{x + |x|}{2}.
$$

PReLU函数, 为ReLU函数的变形, 修改其正部的斜率, 设该斜率$\alpha$为0.01:
$$
    F_{\textmd{act}} = \left\{\begin{array}{ll}
    \alpha x, & x\geq0,\\
    0, & x<0.
    \end{array}\right.
$$

ELU函数, 为ReLU函数的变形, 修改$x<0$处的取值, 设参数$\alpha$为1:
$$
    F_{\textmd{act}} = \left\{\begin{array}{ll}
    x, & x\geq0,\\
    \alpha(e^x-1), & x<0.
    \end{array}\right.
$$

Softplus函数, 将ReLU函数近似光滑化:
$$
    F_{\textmd{act}} = \log(1+\textmd{e}^x).
$$

Sigmoid函数, 即为S型曲线函数:
$$
    F_{\textmd{act}} = \frac{1}{1+\textmd{e}^{-x}}.
$$


\subsection{输出函数}
最基本的输出函数即为线性输出, 即:
$$
    F_{\textmd{out}}(x) = Wx.
$$
其中, $x\in\mathbb{R}^{N_r}$, 
$W$为与$W_{\textmd{out}}$同样大小的矩阵, 本小节其他矩阵也是同样维度. 

还有二次函数也可以作为输出函数:
$$
    F_{\textmd{out}}(x) = W_1x + W_2x^2.
$$

本文引入更多类型的输出函数, 比如具有周期性, 且具有一定有界性的输出函数:
$$
    F_{\textmd{out}}(x) = W_1\sin(x) + W_2\cos(x).
$$


\section{训练阶段和预测阶段}
储备池计算的使用包括两个阶段, 第一个是训练阶段, 即输入数据使储备池中神经元进行演化, 
其中需要确定神经元的迭代规则, 并将每个时刻神经元的状态记录下来; 训练时间结束后, 
对这段时间内神经元状态的时空矩阵进行类似线性回归的操作, 拟合相对应时间内的输出矩阵$O$, 
从而得到输出矩阵以及输出函数; 第二个是预测阶段, 此时每个时刻神经元状态迭代结束后, 
利用输出层输出下一个时刻的输出, 再将其作为下个时刻输入数据, 从而达成自动演化, 
为未来动力学现象的演化提供预测. 

\subsection{训练阶段}
神经元在离散时间节点上的迭代规则定义如下:
$$
    r(t+\Delta t) = F_{\textmd{act}}[W_ii(t)+W_rr(t)],
$$

其中, $i(t)$作为输入变量实时输入, 因此上述完全定义了$r(t)$在训练阶段的演化方程. 
训练阶段储备池是没有输出的, 当训练阶段结束后, 得到了神经元状态的时空矩阵
$R=[r(0), \dots, r(T)]\in\mathbb{R}^{{N_r}\times{n}}$, 以及相对应的输入时空矩阵
$I=[i(0), \dots, i(T)]\in\mathbb{R}^{{N_i}\times{n}}$. 接下来需要确定输出函数, 然后通过拟合得到输出矩阵. 

以二次函数作为输出函数为例, 那么需要得到$W_1, W_2$, 使得下式成立:
$$
    I = W_1R+W_2R^2,
$$
其中, 平方运算是针对矩阵元素的, $W_1, W_2\in\mathbb{R}^{{N_i}\times{N_r}}$.

在采用岭回归(Ridge Regression)进行线性回归之前, 需要先对$W_1, W_2, R$进行一些说明. 
不妨假设$W_1, W_2$的列向量中都只有一半为非零向量, 而且同样位置的列向量两个中只有一个非零, 
这样使得有一半的神经元将会在输出函数中永远表现为线性输出, 而另一部分神经元永远表现为平方输出, 
这样我们将平方输出的神经元对应的状态进行平方后, 再重新组成修改后的状态时空矩阵$\tilde{R}$, 即
$$
    \tilde{R}(i, j) = \left\{\begin{array}{ll}
        R(i, j), & \textmd{如果 $W_1(i)$ 非零},\\
        R(i, j)^2, & \textmd{如果 $W_2(i)$ 非零},
        \end{array}\right.
$$
其中, $W_1(i)$, $W_2(i)$表示为$W_1, W_2$的第$i$个列向量. 

采用岭回归对方程
$$
    I=W\tilde{R},
$$
进行求解, 解可以写出具体表达式:
$$
    W=(I\tilde{R}^T)(\tilde{R}\tilde{R}^T+\beta E),
$$
其中, $E\in\mathbb{R}^{{N_r}\times{N_r}}$是单位阵, $\beta$是正则化参数, 再利用
$$
    W=W_1+W_2,
$$
就可以解得$W_1, W_2$. 从而就将输出函数的具体表达式完全确定下来, 训练结束.

\begin{figure}[htbp]
    \centering
    \includegraphics[scale=0.38]{2.png}
    \caption{储备池训练阶段示意图.}
\end{figure}

\subsection{预测阶段}
预测阶段储备池计算是自动迭代的系统, 满足:
$$
    r(t+\Delta t) = F_{\textmd{act}}[W_iF_{\textmd{out}}(r(t))+W_rr(t)], 
$$
$$
    o(t) = F_{\textmd{out}}(r(t)),
$$
即根据当前时刻的神经元状态和输入变量得到下一个时刻的神经元状态, 再通过输出函数
得到下一个时刻的输出, 而下一个时刻的输出成为下一个时刻的输入, 从而持续迭代. 
预测阶段示意图如下:
\begin{figure}[htbp]
    \centering
    \includegraphics[scale=0.4]{3.png}
    \caption{储备池预测阶段示意图.}
\end{figure}

\section{储备池并联框架}

根据上述给出的多种激活函数和输出函数形式, 一个直观的想法是将多个储备池并联起来, 
给储备池网络带来多元性, 提供更加丰富的神经元状态方程. 

储备池并联框架的训练和单独储备池类似, 只需要将数据输入到储备池中, 
再根据已经确定好的不同激活函数进行迭代, 从而得到神经元时空状态矩阵. 
不同的是, 最后需要将每个储备池产生的神经元时空状态矩阵按照神经元进行拼接, 
得到一个包含所有神经元状态的时空状态矩阵. 

训练迭代结束后, 根据设定好的统一输出函数形式, 进行训练阶段的拟合, 
假设第$k$个储备池产生的神经元时空状态矩阵为$I_k\in\mathbb{R}^{{N_i}\times{n_k}}$, 
($k=1, \dots, K$).
其中$n_k$为第$k$个储备池神经元个数, 那么拼接后的神经元时空状态矩阵为
$I=[I_1, \dots, I_K]\in\mathbb{R}^{{N_i}\times{\sum_{k=1}^Kn_k}}$. 
再相应的进行线性回归求解, 得到$W_{\textmd{out}}$, 
从而确定输出函数的具体形式, 训练阶段结束. 训练阶段示意图如下:

\begin{figure}[htbp]
    \centering
    \includegraphics[scale=0.4]{13.png}
    \caption{并联框架的训练.}
\end{figure}

随后进行预测阶段, 根据确定好的输出函数, 在预测阶段的每个时刻对所有神经元状态作用
输出函数, 得到输出结果, 再到下一个时刻将输出结果作为输入输回到储备池框架中. 
本质上和单个储备池框架一致. 预测阶段示意图如下.

\begin{figure}[htbp]
    \centering
    \includegraphics[scale=0.4]{14.png}
    \caption{并联框架的预测.}
\end{figure}

\section{误差分析}

预测效果主要利用如下几个误差指标进行分析, 其中$o(t)$表示$t$时刻非线性系统的坐标, 
$\tilde{o}(t)$表示$t$时刻储备池输出的预测坐标. $\bar{o}=\sum_{j=1}^no(j\Delta t)$. 

$$
    \textmd{RMSE} = \sqrt{\frac{1}{n}\sum_{j=1}^n[o(j\Delta t)-\tilde{o}(j\Delta t)]^2};
$$

$$
    \textmd{NRMSE} = \sqrt{\sum_{j=1}^n[o(j\Delta t)-\tilde{o}(j\Delta t)]^2 / 
    \Big\{\sum_{j=1}^n[o(j\Delta t)-\bar{o}]^2\Big\}};
$$

$$
    \textmd{MAPE} = \frac{1}{n}\sum_{j=1}^n\frac{|o(j\Delta t)-\tilde{o}(j\Delta t)|}{|o(j\Delta t)|}. 
$$

总体来说, 上述三个指标在误差分析上表现基本一致, 存在细微的区别. 

在实际模型计算中, 由于储备池是随机生成的, 因此结果会收到随机性的影响, 为了减轻随机性
带来的干扰, 采用重复10-20次实验的方法, 并选择误差指标的中位数作为最终的误差, 
尽量避免因为极端数据影响实验总体结果. 

事实上, 实际计算中还存在模型无法对非线性动力系统预测的情况, 
表现为数值过大溢出, 这一般是由于参数选择的问题, 
在神经元状态迭代的过程中, 数值超过了计算机上限, 因此引入成功率SR作为预测分析指标:
$$
    \textmd{SR} = \frac{n_{\textmd{suc}}}{n_{\textmd{total}}}\times100\%,
$$
其中, $n_{\textmd{total}}$用来表示重复实验的次数, 
$n_{\textmd{suc}}$其中成功预测的次数. 
%%%%%%%%%%%%%%%%%%%%%%%%%%%%%%%%%%%%%%%%%%%%%%%%%%%%%%%%%%%%%%%%%%%%%%%%%%%%%%%%%%

\chapter{储备池计算可塑性}
储备池计算的可塑性体现在于其改变框架设定后, 仍然具有不错的效果, 甚至会产生更好的效果. 
本章根据上述提出的激活函数, 输出函数, 分析对比一些具体非线性混沌动力系统预测问题下其表现, 
对结果进行误差分析. 

\section{非线性动力系统}

本章用来预测的一些常见非线性动力系统有如下, 具体模型计算中的初值服从$[0, 1]^{N_i}$上的均匀分布:

Lorenz系统:
\begin{eqnarray*}
    \dot{x_1}&=&10(x_2-x_1),\\
    \dot{x_2}&=&x_1(28-x_3)-x_2,\\
    \dot{x_3}&=&x_1x_2-8x_3/3. 
\end{eqnarray*}

Roessler系统:
\begin{eqnarray*}
    \dot{x_1}&=&-(x_2+x_3),\\
    \dot{x_2}&=&x_1 + 0.2*x_2,\\
    \dot{x_3}&=&0.2+x_3(x_1-5.7). 
\end{eqnarray*}

Sprott系统:
\begin{eqnarray*}
    \dot{x_1}&=&x_2x_3,\\
    \dot{x_2}&=&x_1-x_2,\\
    \dot{x_3}&=&1-x_1x_2. 
\end{eqnarray*}

藏本模型是用来描述大量振子相互耦合同步的模型, 其可以成为一个高维模型, 
具体模型计算中的$\theta_i$初值服从$[0, 2\pi]^{N_i}$上的均匀分布, 
向储备池中输入$x_i$进行训练预测:
\begin{eqnarray*}
    \dot{\theta_i}&=&\omega_i+\frac{K_i}{N}\sum_{j=1}^N\sin(\theta_j-\theta_i),  \\
    x_i&=&\sin(\theta_i),
\end{eqnarray*}
其中, $i=1, \dots, N_i$. 

\bigskip
接下来我们对上述这些非线性系统模型进行预测, 首先对一些参数进行设定, 
$\Delta t=0.01$, $\beta=1\times10^{-4}$, 从而简化参数选择. 

\section{激活函数的选择}

\subsection{Lorenz系统的预测}
对Lorenz系统进行预测, 假设训练时间长度为50, 总共5000个离散时间点数据, 
预测时间长度为2, 总共200个离散时间点数据. 采用二次函数作为输出函数, 
对每个激活函数进行实验, 对参数$\rho$在$[0, 1]$区间上进行遍历. 

\begin{figure}[htbp]
    \centering
    \includegraphics[scale=0.4]{4.png}
    \caption{Lorenz系统预测的RMSE随谱半径的变化.}
\end{figure}

总体来说, 三个误差指标呈现一致的趋势, 下图以RMSE为例, 横轴为谱半径$\rho$, 纵轴为
相应误差数值. 可以发现在储备池计算中最常使用的激活函数$\tanh$在该预测问题上综合表现最差, 其误差与其他
激活函数的预测误差至少相差一个数量级, Sigmoid函数预测效果仅比$\tanh$好, 其他四个
与ReLU相关的激活函数均表现优势, 去掉$\tanh$和Sigmoid重新作图, 得到下图, 发现总体来看, 
Softplus函数表现最稳定, 四个函数的最优表现都相近, 相互之间没有存在明显优势. 
\begin{figure}[htbp]
    \centering
    \includegraphics[scale=0.4]{5.png}
    \caption{Lorenz系统预测的RMSE随谱半径的变化.}
\end{figure}
\begin{figure}[htbp]
    \centering
    \includegraphics[scale=0.4]{6.png}
    \caption{Lorenz系统预测的NRMSE随谱半径的变化.}
\end{figure}
\begin{figure}[htbp]
    \centering
    \includegraphics[scale=0.4]{7.png}
    \caption{Lorenz系统预测的MAPE随谱半径的变化.}
\end{figure}

\clearpage

\begin{table}[htbp]\centering    
    \caption{各种激活函数预测Lorenz系统的成功率.}
    \begin{tabular}{ccccccc}
    \hline\hline
    激活函数 & ELU & tanh & Softplus & PReLU & ReLU & Sigmoid \\
    \hline
    SR   & 98.25\% & 100.00\% & 97.25\% & 98.25\% & 98.25\% & 100.00\% \\
    RMSE   & 0.055584 & 0.358512 & 0.054747 & 0.053647 & 0.051699 & 0.444234 \\
    NRMSE   & 0.004339 & 0.026071 & 0.004219 & 0.004615 & 0.004616 & 0.036059 \\
    MAPE   & 0.001634 & 0.010019 & 0.001602 & 0.001563 & 0.001454 & 0.011114 \\
    \hline\hline
    \end{tabular}

\end{table}

上表是各种激活函数预测Lorenz系统的成功率, 可以发现总体来说都保持了相当高的成功率, 
其失败的实验也集中在谱半径为1的参数设定下. 因此可以说和ReLU相关的函数在预测Lorenz系统
这个问题上相比于tanh和Sigmoid函数具有明显的优势. 

下图展示了一个预测Lorenz系统的例子, 红实线为Lorenz系统轨迹, 蓝虚线为储备池计算预测轨迹, 
直观上来看二者在一段时间内几乎无差. 
\begin{figure}[htbp]
    \centering
    \includegraphics[scale=0.3]{8.png}
    \caption{Lorenz系统预测.}
\end{figure}

\subsection{Roessler系统的预测}
再对Roessler系统进行预测, 实验发现Roessler系统整体可预测性比Lorenz系统好, 
设置训练时间长度为50, 总共5000个离散时间点数据, 
预测时间长度为25, 总共2500个离散时间点数据. 其他设置相同, 得到下表中误差指标结果, 表中
RMSE, NRMSE, MAPE表示的是最小的误差中位数结果.  

\begin{table}[htbp]\centering
    \caption{各种激活函数预测Roessler系统的误差指标.}
    \begin{tabular}{ccccccc}
    \hline\hline
    激活函数 & ELU & tanh & Softplus & PReLU & ReLU & Sigmoid \\
    \hline
    SR   & 64.00\% & 100.00\% & 68.25\% & 75.50\% & 71.25\% & 100.00\% \\
    RMSE   & 0.703173 & 9.791512 & 0.083055 & 0.530036 & 0.361393 & 3.596421 \\
    NRMSE   & 0.106037 & 1.315140 & 0.010932 & 0.069911 & 0.049901 & 0.467725 \\
    MAPE   & 0.078433& 1.239939 & 0.008683 & 0.059694 & 0.036759 & 0.362219 \\
    \hline\hline
    \end{tabular}
\end{table}

\begin{figure}[htbp]
    \centering
    \includegraphics[scale=0.4]{9.png}
    \caption{Roessler系统预测的RMSE随谱半径的变化.}
\end{figure}

激活函数的总体优劣势和Lorenz系统保持一致, 可见储备池计算原有激活函数设置并不一定是最优设置. 
可以发现此时tanh函数和Sigmoid函数的成功率仍为100\%, 然而其他激活函数的成功率只保持在6成以上, 
成功率主要是由两方面导致的, 其一是tanh函数和Sigmoid函数的有界性, 其他ReLU相关的函数不具有有界性, 
迭代次数较多时数值容易溢出; 其二是对于Roessler预测时长较长, 迭代次数相对于Lorenz函数次数明显增加, 
因此对于无界的激活函数更容易导致数值溢出. 

下图为预测Roessler系统的一个例子, 即使对于较长时间的预测, 其仍保持了一定的准确性. 
% \begin{figure}[ht]
%     \begin{minipage}[t]{0.5\linewidth}
%       \centering
%       \includegraphics[scale=0.4]{10.png}
%       \caption{Roessler系统预测}
%     \end{minipage}%
%     \begin{minipage}[t]{0.5\linewidth}
%       \centering
%       \includegraphics[scale=0.4]{12.png}
%       \caption{Sprott系统预测}
%     \end{minipage}
% \end{figure}

\begin{figure}[htbp]
    \centering
    \includegraphics[scale=0.4]{10.png}
    \caption{Roessler系统预测.}
\end{figure}

\subsection{Sprott系统的预测}
最后再对Sprott系统进行预测, 设置训练时间长度为50, 总共5000个离散时间点数据, 
预测时间长度为15, 总共1500个离散时间点数据. 其他设置相同,
得到如下预测的误差结果. 
\begin{table}[htbp]\centering
    \caption{各种激活函数预测Sprott系统的误差指标.}
    \begin{tabular}{ccccccc}
    \hline\hline
    激活函数 & ELU & tanh & Softplus & PReLU & ReLU & Sigmoid \\
    \hline
    SR   & 46.50\% & 100.00\% & 83.00\% & 57.50\% & 56.50\% & 100.00\% \\
    RMSE   & 1.886267 & 2.336720 & 0.075440 & 1.364760 & 1.507258 & 1.688025 \\
    NRMSE   & 1.015897 & 1.049408 & 0.051015 & 0.885821 & 0.836281 & 0.865029 \\
    MAPE   & 0.741928 & 1.176708 & 0.030699 & 0.704040 & 0.604200 & 0.555001 \\
    \hline\hline
    \end{tabular}
\end{table}
\begin{figure}[htbp]
    \centering
    \includegraphics[scale=0.35]{11.png}
    \caption{Sprott系统预测的RMSE随谱半径的变化.}
\end{figure}

此时发现Softplus激活函数呈现出相对于其他激活函数明显的优势, 并且成功率也保持在8成以上. 
但是总体来说所有激活函数对于谱半径的表现都不敏感, 说明谱半径并不成为影响Sprott系统预测的决定性因素.
下图为预测Sprott系统的一个例子.
\begin{figure}[htbp]
    \centering
    \includegraphics[scale=0.30]{12.png}
    \caption{Sprott系统预测.}
\end{figure}

\clearpage

\section{并联框架预测效果}
基于上述各种激活函数的分析, 自然可以考虑能否采用并联的框架提高预测表现, 分析其中各个子储备池
对应的激活函数不同能否具有提高神经元多元性的能力. 限制所有子储备池的神经元个数总和为2000, 
每个子储备池个数相同, 各个子储备池的谱半径采用向前单个储备池预测时的最优参数. 训练时间和
预测时间都保持和单个储备池的情况一致. 

考虑对Lorenz系统进行预测, 去掉明显劣势的激活函数tanh和sigmoid, 剩下4个激活函数对应的
子储备池神经元个数各为500个, 预测误差结果如下:
\begin{table}[htbp]\centering
    \caption{并联框架预测Lorenz系统的误差指标.}
    \begin{tabular}{ccccccc}
    \hline\hline
    RMSE & NRMSE & MAPE & SR \\
    \hline
    0.066165   & 0.004892 & 0.001891 & 100.00\% \\
    \hline\hline
    \end{tabular}
\end{table}

可以发现该预测误差仅比最差的tanh激活函数好, 比所有构成并联框架成分的激活函数都差, 
只是成功率能保持100\%. 因此如果采用不同的激活函数进行并联, 预测结果较差的激活函数会拖累预测结果较好的激活函数, 
使得在同样维度下, 采用单一的最优激活函数进行迭代的预测效果更好. 

如果都采用最好的Softplus函数作为激活函数, 子储备池神经元个数各为500个, 
预测误差结果如下:
\begin{table}[htbp]\centering
    \caption{并联框架预测Lorenz系统的误差指标.}
    \begin{tabular}{ccccccc}
    \hline\hline
    RMSE & NRMSE & MAPE & SR \\
    \hline
    0.060550 & 0.004907 & 0.001695 & 100.00\% \\
    \hline\hline
    \end{tabular}
\end{table}

可以发现结果仍然没有得到改善, 仍然仅比最差的tanh函数表现要好. 
可以发现并联和弱分类器通过boosting得到强分类器的思想不同, 
不能起到明显改善的作用, 单个储备池的神经元个数起到重要作用. 

\section{输出函数的选择}

在储备池计算可塑性方面, 还可以对输出函数的形式进行调整. 

接下来对Kuramoto模型进行预测, 设Kuramoto模型具有10个变量, 
设置训练时间长度为50, 总共5000个离散时间点数据, 
预测时间长度为30, 总共3000个离散时间点数据. 
$K=0.7$, $\omega_1=\omega_2=0.9$, $\omega_3=\omega_4=\omega_5=0.6$, 
$\omega_6=\omega_7=\omega_8=0.4$, $\omega_9=\omega_{10}=0.1$.  

选取4组不同的输出函数, 分别为$F_{\textmd{out}}^1(x)=W_1x+W_2\cos(x)$, 
$F_{\textmd{out}}^2(x)=W_1x+W_2x^2$, 
$F_{\textmd{out}}^3(x)=W_1\sin(x)+W_2\cos(x)$, 
$F_{\textmd{out}}^4(x)=W_1\sin(x)+W_2x^2$. 

在同样参数设定的情况下, 重复5组实验, 得到如下预测的误差结果. 
\noindent\begin{table}[htbp]\centering
    \caption{各种输出函数预测Kuramoto系统的误差指标.}
    \begin{tabular}{ccccc}
    \hline\hline
    输出函数 & $F_{\textmd{out}}^1x$ & $F_{\textmd{out}}^2$ & $F_{\textmd{out}}^3$ & $F_{\textmd{out}}^4$ \\
    \hline
    SR   & 97.00\% & 11.00\% & 100.00\% & 23.50\% \\
    RMSE   & 0.001993 & 1.891190 & 0.007598 & 1.907929 \\
    NRMSE   & 0.000893 & 0.857577 & 0.003448 & 0.858724 \\
    MAPE   & 0.000987 & 0.702695 & 0.003167 & 0.677313 \\
    \hline\hline
    \end{tabular}
\end{table}

\begin{figure}[htbp]
    \centering
    \includegraphics[scale=0.3]{19.png}
    \caption{Kuramoto系统预测的RMSE随谱半径的变化.}
\end{figure}

其中部分缺失的数据代表模型不能够预测Kuramoto模型, 预测结果不收敛, 发散到无穷远处. 
可见如果采用一般的二次函数参数设定, 其很可能不能够预测Kuramoto模型, 即使能够预测
也难以达到较高的准确度. 我们根据Kuramoto模型的性质, 采用$F_{\textmd{out}}^1$和$F_{\textmd{out}}^3$
作为输出函数的模式, 
使得预测效果较好. 例如下图是利用正余弦函数预测的Kuramoto模型的例子. 
\begin{figure}[htbp]
    \centering
    \includegraphics[scale=0.3]{20.png}
    \caption{Kuramoto系统预测.}
\end{figure}


%%%%%%%%%%%%%%%%%%%%%%%%%%%%%%%%%%%%%%%%%%%%%%%%%%%%%%%%%%%%%%%%%%%%%%%%%%%%%%%%%%

\chapter{储备池耦合同步}
第四章将关注储备池计算耦合同步的内容, 介绍储备池耦合的框架, 
分析多个储备池之间耦合同步的条件, 考虑耦合中噪声的引入对于同步效果的影响. 

\section{储备池的耦合}

\begin{figure}[htbp]
    \centering
    \includegraphics[scale=0.4]{15.png}
    \caption{储备池的耦合.}
\end{figure}

\section{耦合能力}

\section{稳健性分析}

\subsection{噪声的种类}

\subsection{噪声对耦合同步产生的影响}

%%%%%%%%%%%%%%%%%%%%%%%%%%%%%%%%%%%%%%%%%%%%%%%%%%%%%%%%%%%%%%%%%%%%%%%%%%%%%%%%%%

\chapter{不完全可观测下的非线性系统的预测和耦合}
第五章将集中分析在动力系统只具备不完全可观测条件下的储备池计算的预测效果, 
以藏本模型(Kuramoto model)为例, 将其部分变量输入到储备池中, 要求预测系统全局变量, 
对各种设定进行数值实验. 在此之上再结合第四章耦合同步的概念, 分析部分变量的同步
能否引发全局变量的同步, 同样对噪声进行鲁棒性分析, 从而进一步理解储备池耦合同步的条件, 
加深对于非线性动力系统同步现象的认识. 

\section{不完全可观测下的全局预测}
对于一个动力系统来说, 内部的状态变量可能是不可能测量的, 只可能测量一些输出变量, 
其他的变量可能只能事后进行测量, 本文称这样的一个动力系统是不完全可观测的. 

想法是将其部分可观测的变量输入到储备池中, 要求预测系统全局变量. 训练阶段只输入
可观测的变量进入储备池, 训练输出函数时是对全局变量进行训练输出. 
\begin{figure}[htbp]
    \centering
    \includegraphics[scale=0.35]{16.png}
    \caption{不完全可观测下的储备池训练.}
\end{figure}

预测阶段每次输出当前时刻的全局变量后, 也只输出可观测变量重新回到储备池中. 
\begin{figure}[htbp]
    \centering
    \includegraphics[scale=0.35]{17.png}
    \caption{不完全可观测下的储备池预测.}
\end{figure}

以藏本模型为例, 其具体参数与第三章第4节相同, 只是将前5个变量作为可测变量, 
后五个变量不可测量. 结果是完全可以实现全局预测的. 
\noindent\begin{table}[htbp]\centering
    \caption{不同不可观测变量个数条件下的预测Kuramoto系统的误差指标.}
    \begin{tabular}{ccccc}
    \hline\hline
    不可观测变量个数 & RMSE & NRMSE & MAPE & SR \\
    \hline
    0   & 0.001381 & 0.000628 & 0.000635 & 100.00\% \\
    2   & 0.003935 & 0.001757 & 0.001863 & 100.00\% \\
    5   & 0.008696 & 0.003927 & 0.004007 & 100.00\% \\
    \hline\hline
    \end{tabular}
\end{table}

自然发现不可观测变量数量越多, 预测结果相对来说更加不准确, 但是都保持了100\%的成功率, 
这主要是由输出函数的有界性所保证的. 

需要注意的是, 如果可观测变量过于单一, 其不具备识别出其他不可观测变量的能力, 
自然也是不可以实现全局预测. 比如当不可观测变量个数为8时, 储备池只输入了同步后的四条
中的一条轨迹, 机器难以通过这一条轨迹去分辨其他变量的坐标. 例如下图是其中的最好一个预测结果, 
误差仍然比较明显. 
\begin{figure}[htbp]
    \centering
    \includegraphics[scale=0.3]{21.png}
    \caption{8个不可观测变量条件下的Kuramoto系统预测.}
\end{figure}

\clearpage

\section{不完全可观性下的耦合}

\begin{figure}[htbp]
    \centering
    \includegraphics[scale=0.4]{18.png}
    \caption{不完全可观性下的耦合.}
\end{figure}

% \section{噪声的鲁棒性分析}


%%%%%%%%%%%%%%%%%%%%%%%%%%%%%%%%%%%%%%%%%%%%%%%%%%%%%%%%%%%%%%%%%%%%%%%%%%%%%%%%%%

\chapter{总结与展望}

\section{讨论}

\section{总结}
先是奠定了良好的基础, 再在此基础上进行了研究. 

本文将在储备池计算的框架下,对Lorenz系统等混沌系统的预测提出具体的方案,
并分析不同方案的预测效果的差别,从而总结出对于复杂动力学现象的学习方法,
加深对于复杂动力系统的认识,提高对于复杂动力学现象的预测能力。
进一步,论文将展示不同混沌系统的同步结果,分析耦合噪声对同步现象的影响,
发现同步现象的内在规律。
\section{展望}

% \clearpage

% 论文大纲:

% 1. 对储备池网络的研究进行综述,其中涉及到本文的至少有三个方向,
% 第一个是一些具体的设置方向,例如储备池的具体设置(稀疏性)等;
% 第二个是一些因为部分可观性,引发的预测;
% 第三个是储备池之间同步的问题,这里面当然还涉及到很多其他的同步话题。

% \bigskip

% 2. 先从介绍各种非线性动力网络出发,
% 介绍储备池网络的设置,包括储备池本身的性质,激活函数,输出函数,
% 训练阶段和预测阶段如何进行,结果的误差分析,
% 最后再加上一些继承性的储备池网络的介绍。

% \bigskip

% 3. 具体的储备池网络设计部分,
% 先从各种激活函数的对比实验开始,再到Kuramoto模型的输出函数的选择,
% 最后到储备池并联的网络设计,观察其是否能够有boosting的效果。

% \bigskip

% 4. 耦合同步的内容,
% 即首先介绍储备池的耦合同步,分析其能否耦合同步,以及对于噪声的鲁棒性;

% 部分可观性引发的预测,
% 即只观察到部分变量,看能否预测全局变量。这一块感觉创新型有待提高。
% 一点小小的创新在于可以预测不同频率的周期性方程。

% 再借此引发结合考虑,即部分可观性的耦合是否也能否耦合,具体思路是:
% 训练了两个部分可观性的储备池,都是输入部分变量,预测全局变量,
% 然后将两个储备池的部分部分变量输出进行耦合,判断其预测的全局变量是否也能够耦合。

% 其具体实践意义可以这么理解,控制部分变量,是的其他不能观察到的变量也能够被控制。

% \textcolor{red}{最后看看能不能和预测再挂上钩, 基于耦合的预测能力, 短长期预测效果。非线性系统的性质(Lyapunov),并分析不同混沌系统耦合后的同步现象。}

%%%%%%%%%%%%%%%%%%%%%%%%%%%%%%%%%%%%%%%%%%%%%%%%%%%%%%%%%%%%%%%%%%%%%%%%%%%%%%%%%%

\begin{thebibliography}{99}
\footnotesize

\bibitem{VKMF} V. Krasnopolsky and M. Fox-Rabinovitz, 
Complex hybrid models combining deterministic and machine learning components for numerical climate modeling and weather prediction,
\emph{Neural Networks}, 2006, 19(2): 122-134.

\bibitem{YTJK} Y. Tang , J. Kurths , W. Lin , E. Ott and L. Kocarev,
Introduction to Focus Issue: When machine learning meets complex systems: Networks, chaos, and nonlinear dynamics,
\emph{Chaos}, 2020, 30(6): 063151.

\bibitem{WSJC} W. Shi, J. Cao, Q. Zhang, Y. Li and L. Xu,
Edge Computing: Vision and Challenges,
\emph{IEEE Internet of Things Journal}, 2016, 3(5): 637-646.

\bibitem{QZHM} Q. Zhu, H. Ma and W. Lin,
Detecting unstable periodic orbits based only on time series: When adaptive delayed feedback control meets reservoir computing,
\emph{Chaos}, 2019, 29: 093125.

\bibitem{LPTC} L. Pecora and T. Carroll,
Synchronization in chaotic systems, 
\emph{Phys. Rev. Lett.}, 1990, 64(8): 821. 

\bibitem{MRAP} M. Rosenblum, A. Pikovsky and J. Kurths,
Phase synchronization of chaotic oscillators, 
\emph{Phys. Rev. Lett.}, 1996, 76(11): 1804-7. 

\bibitem{WMTN} W. Maass, T. Natschläger and H. Markram,
Real-time computing without stable states: a new framework for neural computation based on perturbations, 
\emph{Neural Computation}, 2002, 14(11): 2531-2560. 

\bibitem{HJ} H. Jaeger, 
The ``echo state'' approach to analysing and training recurrent neural networks, in
\emph{Technical Report GMD Report 148, German National Research Center for Information Technology}, 
2001.

\bibitem{BSDV} B. Schrauwen, D. Verstraeten and J. V. Campenhout, 
An overview of reservoir computing: theory, applications and implementations, in
\emph{European Symposium on 15th European Symposium on Artificial Neural Networks. Bruges, Belgium}, 
2007: 471-482.

\bibitem{JPBH} J. Pathak, B. Hunt, M. Girvan, Z. Lu and E. Ott,
Model-Free Prediction of Large Spatiotemporally Chaotic Systems from Data: A Reservoir Computing Approach, 
\emph{Phys. Rev. Lett.}, 2018, 120(2): 024102.

\bibitem{JFJJ} J. Fan, J. Jiang, C. Zhang, X. Wang and Y. Lai,
Long-term prediction of chaotic systems with machine learning, 
\emph{Phys. Rev. Research}, 2020, 2(1): 012080.


\bibitem{AGAP} A. Griffith, A. Pomerance and D. J. Gauthier,
Forecasting chaotic systems with very low connectivity reservoir computers,
\emph{Chaos}, 2019, 29(12): 123108.

\bibitem{TLC} T. L. Carroll,
Using reservoir computers to distinguish chaotic signals, 
\emph{Phys. Rev. E}, 2018, 98:052209.

\bibitem{GBMC} G. Morales, C. Mirasso and M. Soriano,
Unveiling the role of plasticity rules in reservoir computing,
\emph{Neurocomputing}, 2021, 461: 705-715.

\bibitem{JMYW} J. Moon, Y. Wu and W. Lu,
Hierarchical architectures in reservoir computing systems,
\emph{Neuromorphic Computing and Engineering}, 2021, 1: 014006.

\bibitem{WASB} W. A. S. Barbosa, A. Griffith, G. E. Rowlands, L. C. G. Govia, G. J. Ribeill, M. Nguyen, T. A. Ohki and D. J. Gauthier,
Symmetry-aware reservoir computing,
\emph{Phys. Rev. Lett.}, 2021, 104: 045307.

\bibitem{TWHY} T. Weng, H. Yang, C. Gu, J. Zhang and M. Small,
Synchronization of chaotic systems and their machine-learning models,
\emph{Phys. Rev. E}, 2019, 99(4): 042203.

\bibitem{XCTW} X. Chen, T. Weng, C. Gu and H. Yang,
Synchronizing hyperchaotic subsystems with a single variable: A reservoir computing approach,
\emph{Phys. A: Statistical Mechanics and its Applications}, 2019, 534: 122273.

\bibitem{SSAM} S. Saha, A. Mishra, S. Ghosh, S. K. Dana and C. Hens,
Predicting bursting in a complete graph of mixed population through reservoir computing, 
\emph{Phys. Rev. Research}, 2020, 2(3): 033338.

\bibitem{WHYZ} W. Hu, Y. Zhang, R. Ma, Q. Dai, J. Yang,
Synchronization between two linearly coupled reservoir computers, 
\emph{Chaos, Solitons and Fractals}, 2022, 157: 111882.

\end{thebibliography}

\addcontentsline{toc}{chapter}{参考文献}

%%%%%%%%%%%%%%%%%%%%%%%%%%%%%%%%%%%%%%%%%%%%%%%%%%%%%%%%%%%%%%%%%%%%%%%%%%%%%%%%%%

\chapter*{\heiti{致谢}}

% 请对帮助过你完成论文的老师、同学致谢. 也可以在此对您四年大学生活有重要帮助的人致谢.

% 学位论文的致谢在最后一页,并宜单独成页. 

\addcontentsline{toc}{chapter}{致谢}

%%%%%%%%%%%%%%%%%%%%%%%%%%%%%%%%%%%%%%%%%%%%%%%%%%%%%%%%%%%%%%%%%%%%%%%%%%%%%%%%%%

\end{document}
